%%%%%%%%%%%%%%%%%%%%%%%%%%%%%%%%%%%%%%%%%
% Short Sectioned Assignment
% LaTeX Template
% Version 1.0 (5/5/12)
%
% This template has been downloaded from:
% http://www.LaTeXTemplates.com
%
% Original author:
% Frits Wenneker (http://www.howtotex.com)
%
% License:
% CC BY-NC-SA 3.0 (http://creativecommons.org/licenses/by-nc-sa/3.0/)
%
%%%%%%%%%%%%%%%%%%%%%%%%%%%%%%%%%%%%%%%%%

%----------------------------------------------------------------------------------------
%	PACKAGES AND OTHER DOCUMENT CONFIGURATIONS
%----------------------------------------------------------------------------------------

\documentclass[paper=a4, fontsize=11pt]{scrartcl} % A4 paper and 11pt font size

\usepackage[T1]{fontenc} % Use 8-bit encoding that has 256 glyphs
\usepackage{fourier} % Use the Adobe Utopia font for the document - comment this line to return to the LaTeX default
\usepackage[english]{babel} % English language/hyphenation
\usepackage{amsmath,amsfonts,amsthm} % Math packages
\usepackage{amssymb}
\newtheorem{theorem}{Claim}
\usepackage{graphicx}
\usepackage{lipsum} % Used for inserting dummy 'Lorem ipsum' text into the template
\usepackage{listings}
\usepackage{fancyvrb} 
\usepackage{bm} 
\usepackage{xcolor} 
\xdefinecolor{gray}{rgb}{0.4,0.4,0.4} 
\definecolor{LightCyan}{rgb}{0.88,1,1}
\xdefinecolor{blue}{RGB}{58,95,205}% R's royalblue3; #3A5FCD
\usepackage{titlesec}
\newcommand{\sectionbreak}{\clearpage}

\lstset{% setup listings 
        language=R,% set programming language 
        basicstyle=\ttfamily\small,% basic font style 
        keywordstyle=\color{blue},% keyword style 
        commentstyle=\color{red},% comment style 
        numbers=left,% display line numbers on the left side 
        numberstyle=\scriptsize,% use small line numbers 
        numbersep=10pt,% space between line numbers and code 
        tabsize=3,% sizes of tabs 
        showstringspaces=false,% do not replace spaces in strings by a certain character 
        captionpos=b,% positioning of the caption below 
        breaklines=true,% automatic line breaking 
        escapeinside={(*}{*)},% escaping to LaTeX 
        fancyvrb=true,% verbatim code is typset by listings 
        extendedchars=false,% prohibit extended chars (chars of codes 128--255) 
        literate={"}{{\texttt{"}}}1{<-}{{$\bm\leftarrow$}}1{<<-}{{$\bm\twoheadleftarrow$}}1 
        {~}{{$\bm\sim$}}1{<=}{{$\bm\le$}}1{>=}{{$\bm\ge$}}1{!=}{{$\bm\neq$}}1{^}{{$^{\bm\wedge}$}}1,% item to replace, text, length of chars 
        alsoletter={.<-},% becomes a letter 
        alsoother={$},% becomes other 
        otherkeywords={!=, ~, $, \&, \%/\%, \%*\%, \%\%, <-, <<-, /},% other keywords 
        deletekeywords={c}% remove keywords 
} 


\usepackage{sectsty} % Allows customizing section commands
%\usepackage[left=2cm,right=2cm,top=2cm,bottom=3cm]{geometry}
\usepackage[margin=1in]{geometry}
\addtolength{\topmargin}{-.1in}
\allsectionsfont{\normalfont\scshape} % Make all sections centered, the default font and small caps

\usepackage{fancyhdr} % Custom headers and footers
\pagestyle{fancyplain} % Makes all pages in the document conform to the custom headers and footers
\fancyhead{} % No page header - if you want one, create it in the same way as the footers below
\fancyfoot[L]{} % Empty left footer
\fancyfoot[C]{} % Empty center footer
\fancyfoot[R]{\thepage} % Page numbering for right footer
\renewcommand{\headrulewidth}{0pt} % Remove header underlines
\renewcommand{\footrulewidth}{0pt} % Remove footer underlines
\setlength{\headheight}{2pt} % Customize the height of the header

\numberwithin{equation}{section} % Number equations within sections (i.e. 1.1, 1.2, 2.1, 2.2 instead of 1, 2, 3, 4)
\numberwithin{figure}{section} % Number figures within sections (i.e. 1.1, 1.2, 2.1, 2.2 instead of 1, 2, 3, 4)
\numberwithin{table}{section} % Number tables within sections (i.e. 1.1, 1.2, 2.1, 2.2 instead of 1, 2, 3, 4)

\setlength\parindent{0pt} % Removes all indentation from paragraphs - comment this line for an assignment with lots of text

%----------------------------------------------------------------------------------------
%	TITLE SECTION
%----------------------------------------------------------------------------------------

\newcommand{\horrule}[1]{\rule{\linewidth}{#1}} % Create horizontal rule command with 1 argument of height

\title{	
\normalfont \normalsize 
\textsc{University of Texas, McCombs School of Business} \\ [25pt] % Your university, school and/or department name(s)
\horrule{0.1pt} \\[.5cm] % Thin top horizontal rule
\Large Hierarchical modeling of export share panel data \\ % The assignment title
\horrule{.1pt} \\[0cm] % Thick bottom horizontal rule
}
\vspace{10mm}
\subtitle{}

\author{\large Kent Troutman and David Puelz} % Your name

\date{\normalsize\today} % Today's date or a custom date

\begin{document}
\maketitle
\tableofcontents
\newpage


%----------------------------------------------------------------------------------------
%	PROBLEM 1
%----------------------------------------------------------------------------------------

\section{The model}

 
\begin{align}
	y_{ijt} &= \alpha_{i} + \sum_{c=1}^{C}\beta_{ij}^{c}x_{ijt}^{c} + \epsilon_{ijt}
	\\ \nonumber
	\\
	\epsilon_{ijt} &\sim N(0,\sigma^{2})
\end{align}

where $y_{ijt}$ is the response (change in export share) and $\{ x_{ijt}^{c} \}_{c=1}^{C}$ is the set of $C$ covariates.  We need to estimate the following parameters: $\alpha_{i},\beta_{ij}^{c}$.
\\
\\
To form a hierarchical model, we put \textcolor{blue}{priors} and \textcolor{red}{hyperpriors} on the parameters we need to estimate.

\begin{align}
	\color{blue} \beta_{ij}^{c} & \color{blue} \sim N(\mu_{c},\tau_{c}^{2}) \hspace{3mm} \forall c
	\\ \nonumber
	\\
	\color{blue} \alpha_{i} & \color{blue} \sim N(0,\sigma_{\alpha}^{2}) \hspace{3mm} \forall i
	\\ \nonumber
	\\
	\color{blue} \sigma^{2} & \color{blue} \propto \frac{1}{\sigma^{2}}
	\\ \nonumber
	\\
	\color{red} \mu_{c} & \color{red} \sim N(0,\sigma_{0}^{2}) \hspace{3mm} \forall c
	\\ \nonumber
	\\
	\color{red} \tau_{c}^{2} & \color{red} \propto \frac{1}{\tau_{c}} \hspace{3mm} \forall c
	\\ \nonumber
	\\
	\color{red} \sigma_{0}^{2} & \color{red} =  50
\end{align}

\section{Posteriors}

These priors and hyperpriors imply the following posterior distributions of the parameters.  Let's suppose we have $T*I*J$ total observations in our panel.  Here, $T$ is the number of time points, $I$ is the number of country $i$'s and, $J$ is the number of country $j$'s.  We re-write the model is matrix notation as:

\begin{align}
	y = \alpha + X\beta + \epsilon
\end{align}

$y \in \mathbb{R}^{T*I*J}$, $\alpha \in \mathbb{R}^{T*I*J}$ and satisfying $\alpha_{n} = \alpha_{i}$ if $y_{n} = y_{i \cdot \cdot} \hspace{2mm} \forall n \in \{1, \cdots, T*I*J\}$.
\\
\\
 $X \in \mathbb{R}^{(T*I*J) x (I*J*C)}$ is a sparse matrix.
\\
\\ 
$\beta \in \mathbb{R}^{I*J*C}$. 
\\
\\
The complete conditionals are:

\begin{align}
	\beta \vert \cdot &\sim N(m_{1},\Sigma_{1})
	\\
	& \Sigma_{1}^{-1} = \frac{1}{\sigma^{2}}X^{T}X + T^{-1} \nonumber
	\\
	& m_{1} = \Sigma_{1} \left( \frac{1}{\sigma^{2}}  X^{T}( y-\alpha ) + T^{-1} \mu \right) \nonumber
	\\ \nonumber
	\\
	\sigma^{2} \vert \cdot &\sim Ga\left(\frac{T*I*J}{2} + 2 , \hspace{2mm} \frac{(y-\alpha- X\beta)^{T}(y-\alpha- X\beta)}{2}\right)
	\\ \nonumber
	\\
	\tau_{c}^{2} \vert \cdot &\sim Ga\left( \frac{I*J +1}{2} + 1, \hspace{2mm}  \right)
\end{align}









\end{document}